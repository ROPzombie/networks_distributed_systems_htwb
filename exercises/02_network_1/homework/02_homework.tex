%start preamble
\documentclass[paper=a4,fontsize=11pt]{scrartcl}%kind of doc, font size, paper size

\usepackage{fontspec}
\defaultfontfeatures{Ligatures=TeX}
%\setsansfont{Liberation Sans}
\usepackage{polyglossia}	
\setdefaultlanguage[spelling=new, babelshorthands=true]{german}

\usepackage{amsmath}%get math done
\usepackage{amsthm}%get theorems and proofs done
\usepackage{graphicx}%get pictures & graphics done
\graphicspath{{pictures/}}%folder to stash all kind of pictures etc
\usepackage{amssymb}%symbolics for math
\usepackage{amsfonts}%extra fonts
\usepackage []{natbib}%citation style
\usepackage{caption}%captions under everything
\usepackage{listings}
\usepackage[titletoc]{appendix}
\numberwithin{equation}{section} 
\usepackage[printonlyused,withpage]{acronym}%how to handle acronyms
\usepackage{float}%for garphics and how to let them floating around in the doc
\usepackage{wrapfig}%making graphics floated by text and not done by minipage
\usepackage{geometry}
\usepackage{hyperref}
\usepackage{fancyhdr}
\usepackage{menukeys}
\usepackage{xcolor}%nicer colors, here used for links
\usepackage{csquotes}
\usepackage{enumitem}

%settings colors for links
\hypersetup{
    colorlinks,
    linkcolor={blue!50!black},
    citecolor={blue},
    urlcolor={blue!80!black}
}

\definecolor{pblue}{rgb}{0.13,0.13,1}
\definecolor{pgreen}{rgb}{0,0.5,0}
\definecolor{pred}{rgb}{0.9,0,0}
\definecolor{pgrey}{rgb}{0.46,0.45,0.48}

\pagestyle{fancy}
\lhead{NW \& VS -- Übung\\Wintersemester 2021/22}
\rhead{FB 4 -- Angewandte Informatik\\ HTW-Berlin}
\lfoot{Übungsblatt 02 -- Netzwerkinfrastruktur}
\cfoot{}
\fancyfoot[R]{\thepage}
\renewcommand{\headrulewidth}{0.4pt}
\renewcommand{\footrulewidth}{0.4pt}

\lstdefinestyle{Bash}{
  language=bash,
  showstringspaces=false,
  basicstyle=\small\sffamily,
  numbers=left,
  numberstyle=\tiny,
  numbersep=5pt,
  frame=trlb,
  columns=fullflexible,
  backgroundcolor=\color{gray!20},
  linewidth=0.9\linewidth,
  %xleftmargin=0.5\linewidth
}

%%here begins the actual document%%
\newcommand{\horrule}[1]{\rule{\linewidth}{#1}} % Create horizontal rule command with 1 argument of height

\DeclareMathOperator{\id}{id}

\begin{document}
\begin{center}
\Large{\textbf{Übungsblatt 2 -- Netzwerkinfrastruktur}}
\end{center}

\begin{center}\Large{\textbf{Aufgabe A - Wiederholung Vorlesung\\Planung des physischen Netzes}}\end{center}

Sie planen ein kleines Netzwerk, bestehend aus drei bis vier Rechnern (also drei/vier VMs -- im Labor sind dies drei Rechner). Hierfür sollen sie zunächst die Infrastruktur planen.
\begin{enumerate}
	\item Ihr Netzwerk soll aus drei Rechnern bestehen. Diese drei Rechner sind über einen Switch verbunden. In der Virtualisierung haben sie keinen physischen Switch, wir konfigurieren lediglich die Virtualisierungsumgebung, sodass das Netzwerk, wie ein Switched Network arbeitet.\\
	\item Wir realisieren eine Stermtopologie. Recherchieren sie kurz, was eine Topologie im Bereich Rechnernetze ist.
	\item Planen Sie die Netzkonfiguration:
	\begin{enumerate}
		\item  \textbf{Wiederholung:} Rekapitulieren sie was eine IP-Adresse ist. Welche Aufgabe haben diese Adressentypen in einem Netzwerk? s. \cite[S. 331ff]{Kurose2012}
		\item Ordnen sie \emph{IP} im OSI-Modell ein!
		\item Momentan werden vor allem \emph{IPv4} und \emph{IPv6} als Netzwerkschichtprotokolle genutzt. Recherchieren sie einige wichtige Unterschiede zwischen \emph{IPv4} und \emph{IPv6}.
		\item  \textbf{Wiederholung:} Rekapitulieren sie was eine Subnetzmaske ist und wofür diese gebraucht wird.
		\item \textbf{Wiederholung:} Wie spielen IP-Adresse und Subnetzmaske zusammen?
		\item Bestimmte IP-Adressbereiche werden nicht ins Internet weitergeleitet, sie werden als private IP-Adressen bezeichnet. Diese Adressen gibt es sowohl unter \emph{IPv4} als auch unter \emph{IPv6} (\emph{IPv6}: nicht global geroutet). Recherchieren sie, welche IP-Adressbereiche nicht ins Internet geroutet werden.
		\item \textbf{Wiederholung:} Wählen Sie beispielhaft eine Netzwerkadresse (IP-Addresse -- ip address) und Subnetzmaske (subnet mask) für einen möglichst kleinen IP-Adressbereich, die genau für ihre Anzahl Rechner ausreicht.\\
		Wie sähe die Subnetzmaske für $7, 23, 42$ oder $72$ Rechner aus? 
		\item Ich habe im Moodle-Kurs Videos bereitgestellt, die zeigen, wie sie ein virtuelles Netzwerk in virtualBox konfigurieren.\\
		Falls sie nicht im Labor arbeiten, können sie diese als Grundlage des Netzwerkes nehmen.
		\item Die IPv4-Range für das geswitchte Netzwerk ist $172.16.0.0/24$. Wie viele Maschinen könnten sie untergebracht werden? 
		\item Die IPv6-Range für das geswitchte Netzwerk lautet: $fd8a:929:3e98:563c::/64$. Wie viele Maschinen könnten sie untergebracht werden?
		\item Sie sollen jedoch Ihre Netzwerke minimal planen. Welche Netzadressen und Subnetzmasken müssen Sie in Ihre Skizze eintragen?
	\end{enumerate}
	\item Bis jetzt haben sie alles notwendige, um Rechner innerhalb eines Netzwerks zu verbinden. Sollen Rechnernetze miteinander verbunden werden. D.h. die kleinen
		\item Neben IP-Adressen benötigen wir noch zwei weitere Techniken: Routing und Forwarding.
		\begin{enumerate}
			\item Routing legt die Wegwahl der Pakte durch die Netzwerke fest. D.h. wie kommt ein Paket des Rechners $A$ im Netzwerk $\alpha$ in das Netzwerk $\omega$ zum Zielrechner $Z$.\\
			\item Forwarding ist der aktive Prozess, wenn ein Router ein Paket weiterleitet.
		\end{enumerate}
		\item Was sind die Aufgaben eines Routers. Wie erfolgt, im Groben, die Umsetzung des Routings?
	\item Machen sie sich klar, wie Router und IP-Protokoll zusammenhängen. Welche Aufgabe hat das IP-Protokoll, welche Aufgabe hat der Router.
	\item In welche Schicht des OSI-Modells würden sie einen Router einordnen? (Begründung!)
\end{enumerate}

\begin{center}\Large{\textbf{Aufgabe B -- Tools}}\end{center}
Ziel der nächsten Übung ist es das Netzwerk nicht nur theoretisch, sondern auch praktisch umzusetzen. Daher sollen sie die Nutzung einiger Tools in Erfahrung bringen.\\
Mithilfe der Werkzeugsammlungen \emph{iproute2} sowie \emph{net-tools} wird dies in der Regel unter Linux und Unix-Betriebssystemen bewerkstelligt.
\begin{enumerate}
	\item Recherchieren sie kurz, was die Werkzeuge \emph{sudo} bzw. \emph{doas} leisten.
	\item In Betriebssystemen gibt es verschiedene Hintergrunddienste (Daemons), die die Verwaltung des Systems in Teilen organisieren. Da \emph{freeBSD} das Betriebssystem unser Wahl ist, kommt \emph{System V} zum Einsatz.
	\begin{enumerate}
		\item \emph{System-V/service} verfügt über die Möglichkeit bestimmte Dienste zu starten, stopen, etc. Recherchieren sie wie der entsprechende Befehl lautet. Die Man-Pages oder der Link der Fußnote sind gute Anlaufstellen!
		\footnote{\url{https://www.freebsd.org/doc/handbook/configtuning-rcd.html}}\\
		Notieren sie sich die Syntax Wort für Wort, sowie die Bedeutung jedes Wortes (Tokens). 
		\item Wichtige Dienste für die nächste Laborübung ist der \emph{dhcp} . Notieren Sie sich wie kann ein Dienst: gestartet, gestoppt, ein- und ausgeschaltet werden. Wie kann der aktuelle Status abgefragt werden. In welcher Datei können Dienste persisten (dauerhaft) festgehalten werden.
	\end{enumerate}
	\item Übliche Befehle zum Einrichten von Netzwerkadaptern sind \emph{ifconfig} (BSD \emph{net-tools}) oder auch \emph{ip} aus der Werkzeugsammlung \emph{iproute2}. 
 	\item Wie können sie mit den obigen Werkzeugen die aktuelle IP-Adresskonfiguration in Erfahrung bringen?
	\item Recherchieren und notieren sie sich, wie mithilfe des Befehls \emph{ip addr} Netzwerkadapter(n) eine (oder mehrere) IP-Adressen und Subnetzmasken zugewiesen wird. Analog für \emph{ifconfig}.\footnote{\url{http://linux-ip.net/linux-ip/linux-ip.pdf} Appendix C: S 108}\\
	\item Recherchieren sie, wie die IP-Konfiguration in einer Datei festlegen und speichern können, sodass diese weiterhin nach einem Neustart gültig ist.  \footnote{\url{https://www.freebsd.org/doc/de_DE.ISO8859-1/articles/linux-users/network.html}.}
	\item \emph{ICMP} ist ein Diagnose-Protokoll, dass sie bei der Wartung/Nutzung von Netzwerken unterstützt.
	\item Auf welchen Layer des OSI-Modells ordnen sie \emph{ICMP} ein? Begründen sie ihre Wahl.
	\item Recherchieren sie welchen Hinweis ihnen die Folgenden \emph{ICMP}-Messages geben sollen.
	\begin{enumerate}
		\item Connect: network is unreachable
		\item Destination Host Unreachable
		\item Destination Network Unreachable
		\item keine Antwort auf ein Ping
	\end{enumerate}
	\item Das Kommando \emph{ping} nutzt \emph{ICMP} um die Erreichbarkeit anderer Rechner zu prüfen. Recherchieren sie, wie \emph{ping} hierfür genutzt werden kann.
	\item Für das Routing benötigen wir entsprechende Werkzeuge:
	\begin{enumerate}
		\item Recherchieren sie, welches Tool aus \emph{iproute2} genutzt werden kann um Routen zu setzen. Notieren sie sich die Syntax und was die Parameter bewerkstelligen.
	\item Analog dazu: Wie werden Routen mithilfe der \emph{net-tools} konfiguriert?
	\item \url{https://www.freebsd.org/doc/de_DE.ISO8859-1/books/handbook/network-routing.html} bietet ihnen eine gute Anlaufstelle, wie Router und Routing unter \emph{freeBSD} umgesetzt wird.
	\item Recherchieren sie beispielhaft wie eine persistente Lösung aussähe.
	\item Beim aufsetzen des Netzwerkes kann unterschieden werden zwischen \emph{Gateways} und \emph{Default Gateways}. Recherchieren sie diese Unterscheidung.
\end{enumerate}	 
\end{enumerate}

\begin{center}\Large{\textbf{Aufgabe D -- Laborblatt}}\end{center}
\begin{enumerate}
	\item Lesen sie vorbereitend das Laborübungsblatt einmal komplett durch.
	\item Notieren sie sich alle aufkommenden Fragen und Unklarheiten!
	\item Stellen sie ggf. Fragen (Online oder am Anfang der Übung)! 
\end{enumerate}

\begin{center}
\Large{\textbf{Aufgabe E -- Optinal: Remote Laborübung}}
\end{center}
\vskip0.25in
Falls sie nicht an der kommenden Laborübung teilnehmen können, lösen sie folgende Aufgaben:
\begin{enumerate}
	\item Importieren und klonen sie die benötigten VMs!
	\begin{enumerate}
		\item Optimal: Zwei Netzwerke mit je einer \emph{freeBSD}- und einer Linux-VM. (Falls Ressourcen knapp sind, zwei Netzwerke mit einem \emph{freeBSD} und einem Linux). Nutzen sie die VMs ohne grafische Oberfläche! Bei Bedarf können sie die Ressourcen der VMs weiter herunter- oder heraufsetzen.
		\item Als Router soll das \emph{freeBSD} mit grafischer Oberfläche zum Einsatz kommen.
		\item Konfigurieren sie die Adapter, sodass ihre VMs in den entsprechenden virtuellen Netzwerken arbeiten (Host-Only Network)
		\begin{enumerate}
			\item Tutorial Import unter virtualBox im Moodle
			\item Tutorial Einstellen der Netzwerkadapter im Moodle
			\begin{itemize}
				\item Die VMs die nur Hosts sind haben einen aktiven Adapter für ein Host-Only-Network.
				\item Der Router hat drei aktive Adapter, ein Bridge und zwei Host-Only-Network.Adapter.
			\end{itemize}
		\end{enumerate}
		\item Konfigurieren sie den \emph{freeBSD}-Router. Diese VM hat drei aktive Adapter: 
		\begin{itemize}
			\item Bridge-Mode
			\item Host-Only Network $A$
			\item Host-Only Network $B$
		\end{itemize}
	\end{enumerate}
	\item Lesen sie vorbereitend das Laborübungsblatt! Notieren sie sich bei allen Aufgaben, die ihnen nicht klar sind ihre Fragen. 
	\item Notieren sie sich alle Fragen zu Aufgaben, bei denen sie kein Lösungsansatz haben.
\end{enumerate}

\bibliographystyle{plain}
\bibliography{sources}
\end{document}


