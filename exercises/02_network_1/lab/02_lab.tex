%start preamble
\documentclass[paper=a4,fontsize=11pt]{scrartcl}%kind of doc, font size, paper size

\usepackage{fontspec}
\defaultfontfeatures{Ligatures=TeX}
%\setsansfont{Liberation Sans}
\usepackage{polyglossia}	
\setdefaultlanguage[spelling=new, babelshorthands=true]{german}

\usepackage{amsmath}%get math done
\usepackage{graphicx}%get pictures & graphics done
\graphicspath{{pictures/}}%folder to stash all kind of pictures etc
\usepackage{amssymb}%symbolics for math
\usepackage{amsfonts}%extra fonts
\usepackage{caption}%captions under everything
\usepackage{listings}
\usepackage[titletoc]{appendix}
\usepackage[printonlyused,withpage]{acronym}%how to handle acronyms
\usepackage{float}%for garphics and how to let them floating around in the doc
\usepackage{xcolor}%nicer colors, here used for links
\usepackage{wrapfig}%making graphics floated by text and not done by minipage
\usepackage{dsfont}
\usepackage{geometry}
\usepackage{hyperref}
\usepackage{breakurl}
\usepackage{fancyhdr}
\usepackage{multicol}
\usepackage{tasks}
\usepackage{csquotes}

%settings colors for links
\hypersetup{
    colorlinks,
    linkcolor={blue!50!black},
    citecolor={blue},
    urlcolor={blue!80!black}
}

\definecolor{pblue}{rgb}{0.13,0.13,1}
\definecolor{pgreen}{rgb}{0,0.5,0}
\definecolor{pred}{rgb}{0.9,0,0}
\definecolor{pgrey}{rgb}{0.46,0.45,0.48}

\pagestyle{fancy}
\lhead{NW \& VS -- Übung\\Wintersemester 2021/22}
\rhead{FB 4 -- Angewandte Informatik\\ HTW-Berlin}
\lfoot{Übungsblatt 02 -- Netzwerkinfrastruktur}
\cfoot{}
\fancyfoot[R]{\thepage}
\renewcommand{\headrulewidth}{0.4pt}
\renewcommand{\footrulewidth}{0.4pt}

\lstdefinestyle{Bash}{
  language=bash,
  showstringspaces=false,
  basicstyle=\small\sffamily,
  numbers=left,
  numberstyle=\tiny,
  numbersep=5pt,
  frame=trlb,
  columns=fullflexible,
  backgroundcolor=\color{gray!20},
  %linewidth=0.9\linewidth,
  %xleftmargin=0.5\linewidth
}

%%here begins the actual document%%
\newcommand{\horrule}[1]{\rule{\linewidth}{#1}} % Create horizontal rule command with 1 argument of height

\DeclareMathOperator{\id}{id}

\begin{document}
\begin{center}
\Large{\textbf{Übungsblatt 02 -- Netzwerkinfrastruktur}}\\
\end{center}

\begin{center}\Large{\textbf{Aufgabe A - Setup}}\end{center}
Bevor es richtig losgeht, müssen sie folgende Vorbereitungen treffen.
\begin{enumerate}
	\item Sie benötigen drei VMs/Rechner. Im Labor sind dies drei bis vier Rechner.
	\item Remote: Hierfür sollten sie ein minimales \emph{freeBSD}, ein minimales \emph{Linux} \footnote{Minimal heißt hier: ohne grafische Oberfläche -- Headless} und das \emph{freeBSD} mit grafischer Oberfläche (GUI) bereithalten. \\
Importieren sie die VMs. Hierfür habe ich ein kurzes Video im Moodle-Kurs hinterlegt (s. Hausaufgabenblatt).
	\item Ändern sie die Hostnamen der VMs! Jede VM sollte einen individuellen Namen bekommen. Später empfiehlt es sich die Namen den Funktionalitäten zuzuordnen oder ein festes Namensschema zu nutzen. Den Hostnamen können sie in der Datei \path{/etc/rc.conf} ändern.
\end{enumerate}
	
\begin{center}
\Large{\textbf{Aufgabe B - Anzeige der bestehenden Netzwerkkonfiguration}}
\end{center}
Bevor sie ein eigenes kleines Netzwerk einrichten, sollen sie sich mit den dafür Notwendigen Tools vertraut machen. Daher soll zunächst die bestehende Netzwerkkonfiguration untersucht werden.\\
Eine aktive Netzwerkverbindung ist Voraussetzung für die Kommunikation zwischen Rechnern in einem Netzwerk. Jeder Rechner muss hierfür eine passende IP-Adresse haben mit der andere Rechner bzw. Zwischenknoten im Netz erreichbar sind. \footnote{Wenn sie dem Tutorial gefolgt sind, hat die VM mit grafischer Oberfläche jeweils drei Interfaces. Eines davon hat Zugang zu einem DHCP-Netzwerk. Somit auch eine automatisch zugeordnete IP-Adresse.} Die Rechner bekommen zunächst eine IP-Adresse automatisch -- sind also vorkonfiguriert.\\
\begin{enumerate}
	\item Starten sie \emph{freeBSD} mit grafischer Oberfläche und das Linux Betriebssystem.
	\item Nutzen sie für die nachfolgende Aufgabe beide Tools \emph{ip addr} (Linux) als auch \emph{ifconfig} (\emph{freeBSD}))
	\item Lassen sie sich die aktuelle IP-Adresskonfiguration anzeigen.
	\item Wo finden Sie in der Ausgabe die folgenden Informationen:
	\begin{enumerate}
		\item \emph{MAC}-Adresse der Netzwerkkarte
		\item Aktuelle IP-Adresse des Systems
		\item Subnetzmaske (welches Format wir hier genutzt? Können sie dieses übersetzen?)
		\item Besteht eine aktive Verbindung mit dem Netzwerk -- ist das Gerät aktiv?
	\end{enumerate}
	\item Überprüfen sie, ob ein Netzwerkverbindung besteht. Zum Prüfen können sie folgende Aktionen durchführen:
	\begin{enumerate}
		\item Auf der Kommandozeile einen Rechner mit seinem Namen anpingen (bspw.: \url{mi.fu-berlin.de}).
		\item Ping auf eine IP-Adresse (bspw.: $160.45.117.199$).
		\item Ping auf die IP-Adresse in ihrem Netzwerk. Bspw. lokale Router (oft IP: $192.168.178.1$ oder$192.168.0.1$ ) -- funktioniert die Kommunikation im lokalen Netz (LAN)?
		\item Ping auf die eigene IP-Adresse -- wurde der lokale Netzwerkstack richtig gestartet?
	\end{enumerate}
\end{enumerate}

\begin{center}\Large{\textbf{Aufgabe C -- Umsetzung des statischen Netzwerkes}}\end{center}\vskip0.25in
\textbf{Hinweis: Setzen sie alle Konfigurationen zunächst nicht persistent um. Schreiben sie nichts in die Konfigurationsdateien!}
Setzen sie das aus der Planung hervorgegangene Netzwerk um. Drei/Vier Rechner befinden sich innerhalb eines \emph{LAN}s und sollen miteinander kommunizieren.
\begin{enumerate}
	\item Überlegen sie zunächst auf welchen Adapter das Netzwerk konfiguriert werden soll.
	\item Überprüfen sie, ob auf ihren Rechnern das DHCP eingeschaltet ist. Wie können sie feststellen, ob das \emph{DHCP} auf dem Adapter aktiv ist?
	\item Schalten sie auf allen Rechnern den \emph{DHCP}-Dienst für den Adapter aus.
	\item Vorbereitend für den Adapter jeder VM im LAN:
	\begin{enumerate}
		\item Legen sie eine \emph{IPv4}-Adresse fest. Folgendes Schema soll im Labor angewandt werden: $10.0.X.Y$, wobei $X$ ihrer Bankreihe startend bei 1 annimmt. \footnote{Remote: $172.16.X.X$}.
		\item Ordnen sie der \emph{IPv4}-Adresse einer Subnetzmaske zu. Diese sollte minimal sein, d.h. nur so groß, dass zumindest drei bzw. vier Rechner Platz finden.
		\item Konfigurieren sie den Netzwerkadapter mit den oben genannten Werten!\\
		Achten sie darauf, dass sie das korrekte Gerät konfigurieren! Nutzen sie hierfür die üblichen Tools: \emph{ifconfig} und \emph{ip addr}
	\end{enumerate}
	\item Testen sie, ob ihr Netzwerk funktioniert. Nutzen sie \emph{ping} oder \emph{netcat} um dies zu testen.
	\item Haben ihre Rechner einen Zugang zu anderen Rechnern? Können diese Maschinen außerhalb des LANs oder gar Rechner im Internet erreichen? Halten sie ihre Ergebnisse fest.
	\item Ihre VMs unterstützen auch IPv6. Was müssen sie für ein geswitchtes Netzwerk noch konfigurieren?
	\item Testen sie, ob sich ihre VMs auch via IPv6 erreichen können.
\end{enumerate}

\begin{center}\Large{\textbf{Aufgabe D - Statisches Routing}}\end{center}\vskip0.25in
Setzen sie das aus der Planung hervorgegangene Netzwerk (bzw. die Netzwerke) mit den ihn bekannten Tools um.\\
Ihre Netzwerke bestehen aus mindestens drei Rechner. Optimal sollten zwei Rechner im Netzwerk $A$ und $B$ sein -- also ein minimales \emph{freeBSD} und ein Linux je LAN. Zwischen beiden Netzwerken \enquote{sitzt} der Router (\emph{freeBSD} mit GUI).
\begin{enumerate}
	\item \textbf{Für die Hosts:}\\
	\begin{enumerate}
		\item Bevor sie das Netzwerk umsetzen: Legen sie fest welche Netzwerkadapter zu welchem Netzwerk gehören! Ordnen sie entsprechend den Adaptern den Netzwerken zu.
		\item Wie in der vorigen Übung: Legen sie zu jedem Adapter \emph{IPv4} Adresse und Subnetzmaske fest. Die Netze sollten minimal sein!
		\item Überprüfen sie, ob auf allen Adaptern die für das statische Netzwerk der \emph{DHCP}-Dienst ausgeschaltet ist.
		\item Wählen sie für alle benötigten Adapter die gewählten \emph{IPv4} Adressen und Subnetzmasken. Jeder Host benötigt minimal eine IP-Adresse. Der Router zwei!
		\item Setzen sie die gewählten IP-Adressen auf den Hosts.
		\item Überprüfen sie, ob Rechner innerhalb eines LANs sich bereits erreichen können.
		\item Lassen sie sich die aktuelle Routing-Tabelle anzeigen. Welche Informationen entnehmen sie dieser?
		\item Fahren sie mit der Konfiguration des Routers fort.
	\end{enumerate}
	\item \textbf{Für den Router:}\\
	Der Router benötigt eine etwas andere Konfiguration. 
	\begin{enumerate}
		\item Wie die Hosts benötigt ihr Router IP-Adressen. Für jeden Adapter mindestens eine Adresse samt Subnetzmaske. 
		\item Konfigurieren sie die Adapter des Routers mit IP-Adresse und Subnetzmaske.
		\item Der Router sollte anschließend alle Rechner erreichen können. Andersherum sollte natürlich alle VMs den gemeinsamen Router erreichen können.
		\item Aktivieren sie das Forwarding auf dem Router, sodass Pakete aktiv weitergeleitet werden können.
		\end{enumerate}
		\item Finale: Setzen sie die Einträge im Routing-Table!\\
		Es muss eine Route von Netzwerk $A$ in das Zielnetzwerk $B$ geben. Wo muss eine Konfiguration vorgenommen werden.
		\item Der Router kennt sowohl Netzwerk $A$ als auch Netzwerk $B$, kann also beide Netzwerke erreichen. Die Hosts können den Router erreichen. Wo muss der Routing-Table bearbeitet werden (Host oder Router)?
		\item Eine Route ist wie folgt aufgebaut: Zielnetzwerk $\to$ Router der einen Weg ins Zielnetzwerk kennt. Entsprechend:
		\begin{enumerate}
			\item Tragen sie auf den Rechnern entsprechende Routen ein!
			\item Überprüfen sie, ob sich die Rechner über das LAN hinaus erreichen.
		\end{enumerate}
\end{enumerate}

\end{document}