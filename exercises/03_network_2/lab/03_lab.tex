%start preamble
\documentclass[paper=a4,fontsize=11pt]{scrartcl}%kind of doc, font size, paper size

\usepackage{fontspec}
\defaultfontfeatures{Ligatures=TeX}
%\setsansfont{Liberation Sans}
\usepackage{polyglossia}	
\setdefaultlanguage[spelling=new, babelshorthands=true]{german}
		
\usepackage{amsmath}%get math done
\usepackage{amsthm}%get theorems and proofs done
\usepackage{graphicx}%get pictures & graphics done
\graphicspath{{pictures/}}%folder to stash all kind of pictures etc
\usepackage{amssymb}%symbolics for math
\usepackage{amsfonts}%extra fonts
\usepackage{caption}%captions under everything
\usepackage{listings}
\usepackage[printonlyused,withpage]{acronym}%how to handle acronyms
\usepackage{float}%for garphics and how to let them floating around in the doc
\usepackage{xcolor}%nicer colors, here used for links
\usepackage{wrapfig}%making graphics floated by text and not done by minipage
\usepackage{dsfont}
\usepackage{stmaryrd}
\usepackage{geometry}
\usepackage{hyperref}
\usepackage{fancyhdr}
\usepackage{multicol}
\usepackage{csquotes}
\usepackage{enumitem}

%settings colors for links
\hypersetup{
    colorlinks,
    linkcolor={blue!50!black},
    citecolor={blue},
    urlcolor={blue!80!black}
}

\definecolor{pblue}{rgb}{0.13,0.13,1}
\definecolor{pgreen}{rgb}{0,0.5,0}
\definecolor{pred}{rgb}{0.9,0,0}
\definecolor{pgrey}{rgb}{0.46,0.45,0.48}

%Header & Footers
\pagestyle{fancy}
\lhead{Netzwerke -- Übung\\Sommersemester 2021}
\rhead{FB 4 -- Angewandte Informatik\\Hochschule für Technik und Wirtschft Berlin}
\lfoot{Übungsblatt 03 -- Routed LAN}
\cfoot{}
\fancyfoot[R]{\thepage}
\renewcommand{\headrulewidth}{0.4pt}
\renewcommand{\footrulewidth}{0.4pt}

\lstdefinestyle{Bash}{
  language=bash,
  showstringspaces=false,
  basicstyle=\small\sffamily,
  numbers=left,
  numberstyle=\tiny,
  numbersep=5pt,
  frame=trlb,
  columns=fullflexible,
  backgroundcolor=\color{gray!20},
  linewidth=0.9\linewidth,
  %xleftmargin=0.5\linewidth
  upquote=true,
  columns=fullflexible,
  literate={*}{{\char42}}1
         {-}{{\char45}}1
}


%%here begins the actual document%%
\newcommand{\horrule}[1]{\rule{\linewidth}{#1}} % Create horizontal rule command with 1 argument of height


\DeclareMathOperator{\id}{id}

\title{	
\normalfont \normalsize 
\textsc{Übungsblatt 03}
}

\begin{document}
\begin{center}
\Large{\textbf{Übungsblatt 03 -- Netzwerkgrundlagen}}
\end{center}
\begin{center}\Large{\textbf{Aufgabe A -- Umsetzung des Routings}}\end{center}\vskip0.25in
%\setlist[enumerate, 1]{itemsep=\baselineskip}
Setzen sie das aus der Planung hervorgegangene Netzwerk (bzw. die Netzwerke) mit den ihn bekannten Tools um.\\
Ihre Netzwerke bestehen aus mindestens drei VMs (Fünf VMs wäre optimal). Optimal sollten zwei VMs im Netzwerk $A$ und $B$ sein -- also ein minimales \emph{freeBSD} und ein Linux je LAN. Zwischen beiden Netzwerken \enquote{sitzt} der Router (\emph{freeBSD} mit GUI).
\begin{enumerate}
	\item \textbf{Für die Hosts:}\\
	\begin{enumerate}
		\item Bevor sie das Netzwerk umsetzen: Legen sie fest welche Netzwerkadapter zu welchem Netzwerk gehören! Ordnen sie entsprechend den Adaptern den Netzwerken zu.
		\item Wie in der vorigen Übung: Legen sie zu jedem Adapter IPv4 Adresse und Subnetzmaske fest. Die Netze sollten minimal sein!
		\item Überprüfen sie, ob auf allen Adaptern die für das statische Netzwerk der \emph{DHCP}-Dienst ausgeschaltet ist.
		\item Setzen sie für alle benötigten Adapter die gewählten IPv4 Adressen und Subnetzmasken.
		\item Aktivieren sie die Netzwerkadapter und überprüfen sie, ob VMs innerhalb eines LANs sich bereits erreichen können.
		\item Lassen sie sich die aktuelle Routing-Tabelle anzeigen. Welche Informationen entnehmen sie dieser?
		\item Fahren sie mit der Konfiguration des Routers fort.
		\item Nachdem sie den Router aufgesetzt haben:
		\begin{enumerate}
			\item Tragen sie auf den Hosts entsprechende Routen ein!
			\item Überprüfen sie, ob sich die VMs über das LAN hinaus erreichen.
		\end{enumerate}
	\end{enumerate}
	\item \textbf{Für den Router:}\\
	Der Router benötigt eine etwas andere Konfiguration. 
	\begin{enumerate}
		\item Wie bei den Hosts auch benötigt ihr Router IP-Adressen. Für jeden Adapter mindestens eine Adresse samt Subnetzmaske. 
		\item Konfigurieren sie die Adapter des Routers mit IP-Adresse und Subnetzmaske.
		\item Der Router sollte anschließend alle Rechner erreichen können. Andersherum sollte natürlich alle VMs den gemeinsamen Router erreichen können.
		\item Aktivieren sie das Forwarding auf dem Router, sodass Pakete aktiv weitergeleitet werden können.
		\end{enumerate}
		\item Sie können mithilfe eines kleine Chats testen, ob Pakete tatsächlich auf dem Router ankommen. Mithilfe des Tools \emph{netcat} kann getestet werden, ob Pakete im anderen Netzwerk ankommen. Beide Seiten nutzen das Tool \emph{netcat -- nc}. Das Listing \ref{netcat_server} zeigt die Seite des Servers, dieser Code stellt den Server bereit. Der Client darf sich anschließend mithilfe eines \emph{Sockets (Tupel aus IP-Adresse und Port)} verbinden (s. Listing \ref{netcat_client}). 
		\begin{lstlisting}[style=Bash, language=Bash, label={netcat_server}]
#Server port > 1024 
nc -l -p <port_number> <ip_of_server>
#example
nc -l -p 4711 10.0.0.1
		\end{lstlisting}
		
		\begin{lstlisting}[style=Bash, language=Bash, label={netcat_client}]
#Client 
nc <ip_of_server> <port_number>
#example
nc 10.0.0.1 4711
		\end{lstlisting}
		\item \textbf{Alternativ:} Wenn sie bereits mit Wireshark gearbeitet haben, können sie auch dies benutzen, um festzustellen, ob die Pakete korrekt ankommen. 
	\item \textbf{Router: Uplink}\\
	Alle VMs sollten sich nun erreichen können. Rechner außerhalb dieser Netzwerke können sie wahrscheinlich noch nicht erreichen. Beispielsweise Maschinen im Internet.
	\begin{enumerate}
		\item Schauen sie sich die Routing-Tabellen des Routers und der Hosts an. Welche Informationen können sie diesen entnehmen? Welche Art Routen haben sie gesetzt?
        \item Muss am Router eine Anpassung der Routing-Tabelle vorgenommen werden, so das dieser Rechner außerhalb des Netzes erreicht?
        \item Falls ihre Hosts keine Default-Route haben, welcher Rechner wäre als Default-Gateway sinnvoll? Setzen sie in der Routing-Tabelle entsprechende Routen.
        \item Falls ihr Router noch keiner Rechner außerhalb des Netzes erreichen kann:
        \begin{enumerate}
        		\item Hat ihr Router ein Default-Gateway? Falls nein, welcher Rechner wäre das passende Gateway?
        \end{enumerate}
        \item Setzen sie für den Router eine Default-Route, sodass sie Rechner außerhalb des Netzes erreichen können.
        \item Überprüfen sie, ob sie Rechner außerhalb erreichen können!
     	\begin{enumerate}
     		\item Versuchen sie Rechner innerhalb ihres physischen LANs via IP-Adresse zu erreichen -- bspw. Smartphone, Toaster etc. Die IP-Adressen können sie entweder über den Router oder das Gerät selbst in Erfahrung bringen. Sie können auch die Werkzeuge \emph{arp-ping}, \emph{arp-scan} oder \emph{nmap} nutzen.
     		\item Versuchen sie einen Rechner des Internets via IP-Adresse zu erreichen (bspw.: $1.1.1.1$).
     		\item Versuchen sie einen Rechner des Internets über seinen Domainnamen zu erreichen (bspw. htw-berlin.de)
     	\end{enumerate}
	\end{enumerate}
	\item Da ihre einfachen Hosts wahrscheinlich noch keine Rechner außerhalb des LANs erreichen können, muss der Router nun um eine \emph{NAT}-Funktionalität erweitert werde. Der Router muss neben dem Forwarding die Option des NAT-Gateways spendiert bekommen. Diese muss in der \path{/etc/rc.conf} bereitstehen. Tragen sie die \emph{NAT}-Option ein. Legen sie hier ebenfalls das Interface für \emph{NAT} fest. Es müssen keine speziellen \emph{NAT}-Flags gesetzt werden.
	\item Da wir \emph{NAT} nutzen, muss die Firewall entsprechend gesetzt werden. Sie können entweder \emph{pf} oder die Standard-Firewall nutzen.
	\begin{itemize}
		\item Allgemein: \url{https://docs.freebsd.org/de_DE.ISO8859-1/books/handbook/firewalls.html}
		\item Für \emph{pf}: \url{https://docs.freebsd.org/de_DE.ISO8859-1/books/handbook/firewalls-pf.html}
		\item Für \emph{IPFW}: \url{https://docs.freebsd.org/de_DE.ISO8859-1/books/handbook/firewalls-ipfw.html} Abschnitt 30.4.4 In-Kernel NAT
	\end{itemize}
	\item Prüfen sie erneut, ob ihre VMs ohne grafische Oberfläche Rechner via IP-Adresse außerhalb des LANs, ihres Heimnetzes und des Internets erreichen können.
	\item Konfiguration der \emph{resolv.conf}
	\begin{enumerate}
		\item Wahrscheinlich könnten ihre Hosts noch keine Rechner über den Domainnamen erreichen. Daher müssen sie diese noch konfigurieren. Tragen sie alle notwendigen Einträge in die Datei \path{/etc/resolv.conf} ein. Lesen sie entsprechend hier nach: \url{https://docs.freebsd.org/de_DE.ISO8859-1/books/handbook/configtuning-configfiles.html}
		\item Überprüfen sie, ob sie nun Rechner auch via Domainnamen ansprechen können.
	\end{enumerate}
\end{enumerate}

\begin{center}\Large{\textbf{Aufgabe B -- IPv6}}\end{center}\vskip0.25in
Da \emph{IPv4} ein etwas betagteres Protokoll ist und sie fit für die Zukunft sein sollen, sollen sie abschließend ihr Netzwerk mittels \emph{IPv6} umsetzen. Da \emph{IPv6} eine wesentlich größere IP-Range besitzt ist in der Nachfolgenden Tabelle ein mögliches Adressschema aufgezeigt. Auch hier gilt: \emph{IPv6} hat mehr Adressen, dies sollte sie jedoch nicht dazu verleiten verschwenderisch damit umzugehen!
\begin{table}[H]
\centering
\begin{tabular}{cll}
 &\textbf{Netzwerk 1} & \textbf{Netzwerk 2} \\
 Prefix/L & fd & fd\\
 Global ID & 0da5a0170a & ac26d7d170\\
 Subnet ID &  5fd7 & f78c\\
 Combined/CID & fd0d:a5a0:170a:5fd7::/64 & fdac:26d7:d170:f78c::/64 \\
 IPv6 addresses & fd0d:a5a0:170a:5fd7:xxxx:xxxx:xxxx:xxxx & fdac:26d7:d170:f78c:xxxx:xxxx:xxxx:xxxx
\end{tabular}
\end{table}
Sie sollen im Folgenden ein statisches \emph{IPv6}-Netzwerk umsetzen. Ein Routing außerhalb ihrer lokalen Infrastruktur ist kein muss, da es immer noch Anbieter (oder Hardware) gibt, die keinen \emph{IPv6} unterstützt.
\begin{enumerate}
	\item Adaptieren sie ihren \emph{IPv4}-Netzwerkaufbau auf \emph{IPv6}. D.h. der grundsätzliche Aufbau des Netzwerkes soll nicht verändert werden. Sie fügen den Adaptern lediglich \emph{IPv6}-Adressen hinzu.
	\item Vergeben sie in ihrem Netzwerk entsprechende Adressen. Vergessen sie nicht entsprechende Adressen für den Router zu setzen.
	\item Tragen sie für alle Teilnehmer entsprechende Routen ein.
	\item Denken sie daran das Forwarding für IPv6 zu aktivieren!
	\item Testen Sie Ihre Netzwerke mit \emph{ping6}.
	\item Falls ihr lokale Infrastruktur es zulässt, können sie auch ein Default-Gateway konfigurieren, sodass die VMs via \emph{IPv6} ins Internet kommen.
	\item Müssen sie Einträge in ihre \emph{resolv.conf} ändern? Falls ja: Ändern sie dies entsprechend.
\end{enumerate}

\end{document}