%start preamble
\documentclass[paper=a4,fontsize=11pt]{scrartcl}%kind of doc, font size, paper size

\usepackage{fontspec}
\defaultfontfeatures{Ligatures=TeX}
%\setsansfont{Liberation Sans}
\usepackage{polyglossia}	
\setdefaultlanguage[spelling=new, babelshorthands=true]{german}
			
\usepackage{amsmath}%get math done
\usepackage{amsthm}%get theorems and proofs done
\usepackage{graphicx}%get pictures & graphics done
\graphicspath{{pictures/}}%folder to stash all kind of pictures etc
\usepackage{amssymb}%symbolics for math
\usepackage{amsfonts}%extra fonts
\usepackage{caption}%captions under everything
\usepackage{listings}
\numberwithin{equation}{section} 
\usepackage{float}%for garphics and how to let them floating around in the doc
\usepackage{xcolor}%nicer colors, here used for links
\usepackage{dsfont}
\usepackage{stmaryrd}
\usepackage{geometry}
\usepackage{hyperref}
\usepackage{fancyhdr}
\usepackage{multicol}
\usepackage{csquotes}
\usepackage{enumitem}

\usepackage[backend=biber,style=alphabetic,
citestyle=alphabetic]{biblatex} %biblatex mit biber laden
\addbibresource{sources.bib}

%settings colors for links
\hypersetup{
    colorlinks,
    linkcolor={blue!50!black},
    citecolor={blue},
    urlcolor={blue!80!black}
}

\definecolor{pblue}{rgb}{0.13,0.13,1}
\definecolor{pgreen}{rgb}{0,0.5,0}
\definecolor{pred}{rgb}{0.9,0,0}
\definecolor{pgrey}{rgb}{0.46,0.45,0.48}


\pagestyle{fancy}
\lhead{Netzwerke und verteilte Systeme\\
 Übung SoSe 2022}
\rhead{Informatik in Kultur und Gesundheit\\HTW-Berlin FB 4}
\lfoot{Grundlegende Kommunikation}
\cfoot{}
\fancyfoot[R]{\thepage}
\renewcommand{\headrulewidth}{0.4pt}
\renewcommand{\footrulewidth}{0.4pt}

\lstdefinestyle{Bash}{
  language=bash,
  showstringspaces=false,
  basicstyle=\small\sffamily,
  numbers=left,
  numberstyle=\tiny,
  numbersep=5pt,
  frame=trlb,
  columns=fullflexible,
  backgroundcolor=\color{gray!20},
  linewidth=0.9\linewidth,
  %xleftmargin=0.5\linewidth
}


%%here begins the actual document%%
\newcommand{\horrule}[1]{\rule{\linewidth}{#1}} % Create horizontal rule command with 1 argument of height


\begin{document}
\begin{center}
\Large{\textbf{Übungsblatt 2 -- Grundlegende Kommunikation}}
\end{center}

\begin{center}\Large{\textbf{Aufgabe A -- HTTP(S)}}\end{center}\vskip0.25in
Kein anderes Protokoll ist für das World-Wide-Web so wichtig wie HTTP. In diesem Teil sollen sie recherchieren, wie die bunten Seiten in Ihren Browser kommen.
\begin{enumerate}
	\item Recherchieren sie zunächst was HTTP ist. Eine gute Anlaufstelle wäre \cite[S. 98ff]{Kurose2012}.
	\item Erläutern sie grob, was eine Client-Server-Architektur ist. Wieso entspricht HTTP diesem Modell?
	\item Auf welcher Schicht des OSI-Modells ordnen Sie HTTP ein?
	\item Recherchieren sie was unter einem Port verstanden wird! Ein grobes Verständnis genügt.
	\item Auf welchen Port laufen meistens Webserver? Auf welchem Port läuft die verschlüsselte Variante HTTPs?
	\item HTTP ist ein zustandsloses Protokoll. Erläutern Sie diese Aussage!
	\item HTTP arbeitet mithilfe von Methoden. Erläutern sie kurz folgende Methoden:
	\begin{itemize}
		\item \emph{GET}
		\item \emph{HEAD}
		\item \emph{POST}
		\item \emph{PUT}
		\item \emph{DELETE}
		\item \emph{TRACE}
	\end{itemize}
	\item Machen sie sich kurz klar, welche Aufgabe TLS übernimmt. (Hinweis: An dieser Stelle genügt es, wenn sie wissen was TLS macht.)
	\item Auf welcher Schicht arbeitet TLS? Wenn Sie das Akronym auflösen, sollte die Lösung Ihnen entgegen fallen.
\end{enumerate}





\printbibliography

\end{document}
