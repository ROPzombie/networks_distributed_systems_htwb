%start preamble
\documentclass[paper=a4,fontsize=11pt]{scrartcl}%kind of doc, font size, paper size

\usepackage{fontspec}
\defaultfontfeatures{Ligatures=TeX}
%\setsansfont{Liberation Sans}
\usepackage{polyglossia}	
\setdefaultlanguage[spelling=new, babelshorthands=true]{german}
			
\usepackage{amsmath}%get math done
\usepackage{amsthm}%get theorems and proofs done
\usepackage{graphicx}%get pictures & graphics done
\graphicspath{{pictures/}}%folder to stash all kind of pictures etc
\usepackage{amssymb}%symbolics for math
\usepackage{amsfonts}%extra fonts
\usepackage{caption}%captions under everything
\usepackage{listings}
\numberwithin{equation}{section} 
\usepackage{float}%for garphics and how to let them floating around in the doc
\usepackage{xcolor}%nicer colors, here used for links
\usepackage{dsfont}
\usepackage{stmaryrd}
\usepackage{geometry}
\usepackage{hyperref}
\usepackage{fancyhdr}
\usepackage{multicol}
\usepackage{csquotes}
\usepackage{enumitem}

\usepackage[backend=biber,style=alphabetic,
citestyle=alphabetic]{biblatex} %biblatex mit biber laden
\addbibresource{sources.bib}

%settings colors for links
\hypersetup{
    colorlinks,
    linkcolor={blue!50!black},
    citecolor={blue},
    urlcolor={blue!80!black}
}

\definecolor{pblue}{rgb}{0.13,0.13,1}
\definecolor{pgreen}{rgb}{0,0.5,0}
\definecolor{pred}{rgb}{0.9,0,0}
\definecolor{pgrey}{rgb}{0.46,0.45,0.48}


\pagestyle{fancy}
\lhead{Netzwerke und verteilte Systeme\\
 Übung SoSe 2022}
\rhead{Informatik in Kultur und Gesundheit\\HTW-Berlin FB 4}
\lfoot{Grundlegende Kommunikation}
\cfoot{}
\fancyfoot[R]{\thepage}
\renewcommand{\headrulewidth}{0.4pt}
\renewcommand{\footrulewidth}{0.4pt}

\lstdefinestyle{Bash}{
  language=bash,
  showstringspaces=false,
  basicstyle=\small\sffamily,
  numbers=left,
  numberstyle=\tiny,
  numbersep=5pt,
  frame=trlb,
  columns=fullflexible,
  backgroundcolor=\color{gray!20},
  linewidth=0.9\linewidth,
  %xleftmargin=0.5\linewidth
}


%%here begins the actual document%%
\newcommand{\horrule}[1]{\rule{\linewidth}{#1}} % Create horizontal rule command with 1 argument of height


\begin{document}
\begin{center}
\Large{\textbf{Übungsblatt 2 -- Socket-Kommunikation}}
\end{center}

\begin{center}\Large{\textbf{Aufgabe A -- Grundlagen IP}}\end{center}\vskip0.25in
\begin{enumerate}
	\item Im Labor besteht das Netzwerk aus 20 Rechnern. Diese Rechner sind über einen Switch verbunden. Das heißt alles Rechner können via IP-Protokoll miteinander kommunizieren.\\
	Rekapitulieren sie was eine IP-Adresse ist. Welche Aufgabe haben diese Adressentypen in einem Netzwerk?
	\item Rekapitulieren sie was eine Subnetzmaske ist und wofür diese gebraucht wird.
	\item Wie spielen IP-Adresse und Subnetzmaske zusammen?
	\item Was ist eine Broadcast-Adresse? Wozu wird diese benötigt und wie wird diese gebildet?
	\item Bestimmte IP-Adressbereiche werden nicht ins Internet weitergeleitet, sie werden als private IP-Adressen bezeichnet. Diese Adressen gibt es sowohl unter \emph{IPv4} als auch unter \emph{IPv6} (\emph{IPv6}: nicht global geroutet). Recherchieren sie, welche IP-Adressbereiche nicht ins Internet geroutet werden.
	\item Gegeben sei der Netzbereich $10.23.42.0$. Wie sieht die Netzmaske und Broadcast-Adresse $7, 23, 42$ oder $72$ Rechner aus? 
	\item Wie verschiebt sich die Netzmaske und Broadcast-Adresse der vorigen Aufgabe, wenn der Netzbereich bei $10.23.42.16$ startet?
	\item Angenommen sie bekommen den Netzblock $173.168.0.0/16$.
	\begin{enumerate}
		\item Wie viele Hosts können mit diesem Netz maximal abgedeckt werden?
		\item Wie sieht die Schreibweise des Netzes in Quad-Dot-Notation und binärer Schreibweise aus?
		\item Sie sollen das Netz wie folgt partitionieren: Insgesamt existieren sieben Netzwerke. Zwei gleich große Netzwerke Netzwerke mit Platz für 80 Rechner. Zusätzlich sollen fünf Netzwerke mit je 32 Rechner untergebracht werden.\\
		Schlagen sie eine mögliche Partition des Netzwerkes vor. Geben sie zu jedem Netz Netzadresse, Range und Broadcast-Adresse an.
\end{enumerate}		 
\end{enumerate}

\begin{center} \Large{\textbf{Aufgabe B - Anzeige der bestehenden Netzwerkkonfiguration}} \end{center}

\begin{enumerate}
	\item Nutzen sie für die nachfolgende Aufgabe das Kommando \emph{ifconfig} Lassen sie sich die aktuelle IP-Adresskonfiguration anzeigen.
	\item Mit dem Aufruf von \emph{ifconfig}: Wo finden sie in der Ausgabe die folgenden Informationen:
	\begin{enumerate}
		\item \emph{MAC}-Adresse der Netzwerkkarte
		\item Aktuelle IP-Adresse des Systems
		\item Subnetzmaske (welches Format wir hier genutzt? Können sie dieses übersetzen?)
		\item Besteht eine aktive Verbindung mit dem Netzwerk -- ist das Gerät aktiv?
	\end{enumerate}
	\item Überprüfen sie, ob ein Netzwerkverbindung besteht. Zum Prüfen können sie folgende Aktionen durchführen:
	\begin{enumerate}
		\item Auf der Kommandozeile einen Rechner mit seinem Namen anpingen (bspw.: \url{mi.fu-berlin.de}).
		\item Ping auf eine IP-Adresse (bspw.: $160.45.117.199$).
		\item Ping auf die eigene IP-Adresse -- wurde der lokale Netzwerkstack richtig gestartet?
	\end{enumerate}
	\item Versuchen sie anschließend einen Rechner innerhalb des Netzes zu erreichen. Bspw. den Nachbarrechner.
	\item Was passiert bei einem \emph{ping} auf $192.168.0.1$?
	\item Was passiert wenn sie \emph{ping} auf eine nicht vorhanden IP-Adresse im  Labor absetzen?
	\item Was passiert wenn sie eine Broadcast-Adresse anpingen? Erläutern sie zunächst, was theoretisch passieren müsste. Was passiert, wenn sie das Kommando \emph{ping} auf die Broadcast-Adresse ihres Netzes absetzen?
	\item Welchen Zweck hat das Interface \emph{lo}? Welche IP-Adresse wurde hier hinterlegt? Welche Netz- und Broadcast-Adresse hat dieses Interface?
\end{enumerate}

\begin{center} \Large{\textbf{Aufgabe C - Socket-Kommunikation}} \end{center}
\begin{enumerate}
	\item Rekapitulieren sie, was unter einer Server-Client-Kommunikation verstanden wird. Wie sieht ein typischer Ablauf der Kommunikation aus?
	\item Welche Rolle spielt die IP-Adresse bei dieser Art der Kommunikation?
	\item Neben der IP-Adresse benötigen sie noch eine weitere Information für den Aufbau einer Socket-Verbindung -- den Port. Erläutern sie, was ein Port ist. In welchen Layer des OSI-Modells ordnen sie die Ports ein?
	\item Das Kommando \emph{netcat} erlaubt es beliebige Socket-Kommunikationen aufzubauen. \emph{netcat} benötigt für die Kommunikation eine IP-Adresse und einen Port. Wie legen sie beide Parameter mit \emph{netcat} fest? Rufen sie hierfür die \emph{Man-Page} auf und recherchieren sie.
	\begin{lstlisting}[style=Bash, language=Bash]
man nc
	\end{lstlisting}
	\item Im folgenden soll ein Chat-Server mit \emph{netcat} aufgebaut werden.
	\begin{enumerate}
		\item Im ersten Schritt soll eine lokale Kommunikation ermöglicht werden. Welcher Interface sollte hierfür genutzt werden?
		\begin{enumerate}
			\item Wie sieht das Kommando für den Aufbau des \emph{netcat}-Servers aus? Führen sie das Kommando anschließend aus.
			\item Mit welchem Befehl können sie sich mit dem lokalen \emph{netcat}-Server verbinden? Verbinden sie sich mit ihrem lokalen Server.
		\end{enumerate}
		\item Analog zu vorigen Aufgabe soll der \emph{netcat}-Server nun nicht mehr nur lokal aufrufbar sein, sondern im \emph{LAN} verfügbar sein.
		\begin{enumerate}
			\item Was müssen sie ändern am Kommando des Servers ändern, sodass dieser im LAN ansprechbar wird?
			\item Führen sie die Änderungen durch und starten sie den \emph{netcat}-Server!
			\item Finden sie einen Chat-Server im Labor, den sie via \emph{netcat} erreichen können.
		\end{enumerate}
		\item Nachdem sie sich erfolgreich mit einem Server verbunden haben: Was passiert, wenn mehrere Clients versuchen sich mit dem Server zu verbinden?
	\end{enumerate}
\end{enumerate}




\printbibliography

\end{document}
