%start preamble
\documentclass[paper=a4,fontsize=11pt]{scrartcl}%kind of doc, font size, paper size

\usepackage{fontspec}
\defaultfontfeatures{Ligatures=TeX}
%\setsansfont{Liberation Sans}
\usepackage{polyglossia}	
\setdefaultlanguage[spelling=new, babelshorthands=true]{german}
			
\usepackage{amsmath}%get math done
\usepackage{amsthm}%get theorems and proofs done
\usepackage{graphicx}%get pictures & graphics done
\graphicspath{{pictures/}}%folder to stash all kind of pictures etc
\usepackage{amssymb}%symbolics for math
\usepackage{amsfonts}%extra fonts
\usepackage{caption}%captions under everything
\usepackage{listings}
\numberwithin{equation}{section} 
\usepackage{float}%for garphics and how to let them floating around in the doc
\usepackage{xcolor}%nicer colors, here used for links
\usepackage{dsfont}
\usepackage{stmaryrd}
\usepackage{geometry}
\usepackage{hyperref}
\usepackage{fancyhdr}
\usepackage{multicol}
\usepackage{csquotes}
\usepackage{enumitem}
\usepackage{pythonhighlight}

\usepackage[backend=biber,style=alphabetic,
citestyle=alphabetic]{biblatex} %biblatex mit biber laden
\addbibresource{sources.bib}

%settings colors for links
\hypersetup{
    colorlinks,
    linkcolor={blue!50!black},
    citecolor={blue},
    urlcolor={blue!80!black}
}

\definecolor{pblue}{rgb}{0.13,0.13,1}
\definecolor{pgreen}{rgb}{0,0.5,0}
\definecolor{pred}{rgb}{0.9,0,0}
\definecolor{pgrey}{rgb}{0.46,0.45,0.48}


\pagestyle{fancy}
\lhead{NW+VS -- Übung\\WiSe 2021/22}
\rhead{FB 4 -- IKG \\ HTW-Berlin}
\lfoot{Übungsblatt 03}
\cfoot{}
\fancyfoot[R]{\thepage}
\renewcommand{\headrulewidth}{0.4pt}
\renewcommand{\footrulewidth}{0.4pt}

\lstdefinestyle{Bash}{
  language=bash,
  showstringspaces=false,
  basicstyle=\small\sffamily,
  numbers=left,
  numberstyle=\tiny,
  numbersep=5pt,
  frame=trlb,
  columns=fullflexible,
  backgroundcolor=\color{gray!20},
  linewidth=0.9\linewidth,
  %xleftmargin=0.5\linewidth
}

\lstdefinestyle{Bash}{
  language=bash,
  showstringspaces=false,
  basicstyle=\small\sffamily,
  numbers=left,
  numberstyle=\tiny,
  numbersep=5pt,
  frame=trlb,
  columns=fullflexible,
  backgroundcolor=\color{gray!20},
  linewidth=0.9\linewidth,
  %xleftmargin=0.5\linewidth
}

% Python style for highlighting
\newcommand\pythonstyle{\lstset{
language=Python,
basicstyle=\ttm,
morekeywords={self},              % Add keywords here
keywordstyle=\ttb\color{deepblue},
emph={MyClass,__init__},          % Custom highlighting
emphstyle=\ttb\color{deepred},    % Custom highlighting style
stringstyle=\color{deepgreen},
frame=tb,                         % Any extra options here
showstringspaces=false
}}


%%here begins the actual document%%
\newcommand{\horrule}[1]{\rule{\linewidth}{#1}} % Create horizontal rule command with 1 argument of height

\begin{document}
\begin{center}
\Large{\textbf{Übungsblatt 03 -- Einfache verteilte System mit Sockets und HTTP}}
\end{center}

\begin{center}\Large{\textbf{Aufgabe A -- Scapy}}\end{center}\vskip0.25in
Eine gute Anlaufstelle für die Benutzung von \emph{Scapy} finden sie unter: \url{https://scapy.readthedocs.io/en/latest/introduction.html}.
\begin{enumerate}
	\item Starten sie eine interaktive \emph{Scapy}-Session mithilfe des Terminals:
	\begin{lstlisting}[style=Bash, language=Bash]
scapy
	\end{lstlisting}
	Diese funktioniert analog zur \emph{ipython} Shell der letzten Übung.
	\begin{enumerate}
		\item Welche \emph{HTTP}-Protokolle und Funktionalitäten unterstützt \emph{Scapy}?
		\begin{python}
>>> load_layer("http")
>>> explore(scapy.layers.http)
		\end{python}
		\item Lassen sie in \emph{Scapy} anzeigen, wie eine Default-HTTP-Request aussieht.
		\item Lassen sie in \emph{Scapy} anzeigen, wie eine Default-HTTP-Response aussieht.
	\end{enumerate}		
	\item Wie können sie einzelne Layer (also Protokolle) laden?
	\item Laden sie den \emph{http}-Layer.
	\item Lesen sie folgenden \emph{HTTP}-Request:
	\begin{python}
load_layer("http")
req = HTTP()/HTTPRequest(
    Accept_Encoding=b'gzip, deflate',
    Cache_Control=b'no-cache',
    Connection=b'keep-alive',
    Host=b'HERE.WEB.URL',
    Pragma=b'no-cache'
)
a = TCP_client.tcplink(HTTP, "HERE.WEB.URL", PORT)
answser = a.sr1(req)
a.close()
with open("HERE.WEB.URL", "wb") as file:
    file.write(answser.load)
\end{python}
	\item Passen sie den Code an, sodass sie eine \emph{HTTP}-Anfrage an den Webserver der \emph{HTW-Berlin} stellen können (\url{www.htw-berlin.de}).
\end{enumerate}

\begin{center}\Large{\textbf{Aufgabe B -- Chat-Server}}\end{center}\vskip0.25in

\begin{enumerate}
	\item Unter folgender \emph{URL} finden sie den Code für den Chat-Server: \url{https://pastebin.com/raw/5FeQvUnZ}.
	\begin{enumerate}
		\item Laden sie den Code mittels \emph{wget} herunter.
		\textbf{Hinweis:}
		\begin{lstlisting}[style=Bash, language=Bash]
wget -O server.py URL
		\end{lstlisting}
		\item Nachdem sie in den Hausaufgaben den Code analysiert haben, soll dieser nun auf der VM ausgeführt werden.\\
		Öffnen sie den Server-Code!
		\item Mithilfe des Befehls \emph{ifconfig em0} können sie die aktuelle IP-Adresse des Rechners in Erfahrung bringen. Notieren sie sich diese und tragen sie diese im Code des Servers ein.
		\item Welche Ports kommen prinzipiell in Betracht für den Server? M.a.W. für welche Ports benötigen sie spezielle Rechte und für welche nicht?
		\item Legen sie im Code den Port für ihren Server fest.
		\item Im Code ist noch kein Interpreter festgelegt worden. Tragen sie in die erste Zeile des Code folgende Code ein:
		\begin{python}
#!/usr/bin/env python3
		\end{python}
		Diese Zeile legt fest, dass wir Python als Interpreter für die Ausführung nutzen möchten.
		\item Ändern sie die Zeile
		\begin{python}
from threading import *
		\end{python}
		in
		\begin{python}
import threading
		\end{python}
		Andernfalls wird der Namespace für das Multithreading nicht korrekt aufgelöst.
		\item Das Python-Script soll auf unserer VM ausgeführt werden. Hierfür benötigt es entsprechende Ausführungsrechte. Das Kommando \emph{chmod} bietet diese Möglichkeit:\\
		\begin{lstlisting}[style=Bash, language=Bash]
chmod u+x server.py
		\end{lstlisting}
		Geben sie dem Server Ausführungsrechte!
		\item Testen sie anschließend, ob ihr Server tatsächlich funktioniert! Bricht das Python-Script an einer Stelle ab, müssen wir eine Fehlerursachenfindung betreiben.
	\end{enumerate}
	\item Unter folgender \emph{URL} finden sie den Code für den Chat-Client: \url{https://pastebin.com/raw/UBU0Yzj2}.
	\begin{enumerate}
		\item Laden sie den Code mittels \emph{wget} herunter.
		\textbf{Hinweis:}
		\begin{lstlisting}[style=Bash, language=Bash]
curl -O client.py URL
		\end{lstlisting}
		\item Nachdem sie in den Hausaufgaben den Code analysiert haben, soll dieser nun auf der VM ausgeführt werden.\\
		Öffnen sie den Client-Code!
		\item In diesem Script müssen ebenfalls IP-Adresse und Port für den Socket angegeben werden. Wozu werden diese Angaben benötigt?
		\item Welche Angaben müssen hier hinterlegt werden?
		\item Legen sie im Code den Port für ihren Server fest.
		\item Im Code ist noch kein Interpreter festgelegt worden. Tragen sie in die erste Zeile des Code folgende Code ein:
		\begin{python}
#!/usr/bin/env python
		\end{python}
		Diese Zeile legt fest, dass wir Python als Interpreter für die Ausführung nutzen möchten.
		\item Das Python-Script soll auf unserer VM ausgeführt werden. Hierfür benötigt es entsprechende Ausführungsrechte. Das Kommando \emph{chmod} bietet diese Möglichkeit:\\
		\begin{lstlisting}[style=Bash, language=Bash]
chmod u+x client.py
		\end{lstlisting}
		Geben sie dem Server Ausführungsrechte!
		\item Testen sie anschließend, ob ihr Client tatsächlich funktioniert! Kann sich der Client mit dem Server verbinden?
		\item Sobald ihr Chat-System funktioniert können sie auch weitere Clients sich auf den Server Verbinden. Wenn sie dem Server eine öffentliche IP-Adresse gegeben haben, können auch andere VMs diese Server erreichen. Testen sie dies!
	\end{enumerate}
\end{enumerate}

\end{document}
