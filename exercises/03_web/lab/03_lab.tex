%start preamble
\documentclass[paper=a4,fontsize=11pt]{scrartcl}%kind of doc, font size, paper size

\usepackage{fontspec}
\defaultfontfeatures{Ligatures=TeX}
%\setsansfont{Liberation Sans}
\usepackage{polyglossia}	
\setdefaultlanguage[spelling=new, babelshorthands=true]{german}
			
\usepackage{amsmath}%get math done
\usepackage{amsthm}%get theorems and proofs done
\usepackage{graphicx}%get pictures & graphics done
\graphicspath{{pictures/}}%folder to stash all kind of pictures etc
\usepackage{amssymb}%symbolics for math
\usepackage{amsfonts}%extra fonts
\usepackage{caption}%captions under everything
\usepackage{listings}
\numberwithin{equation}{section} 
\usepackage{float}%for garphics and how to let them floating around in the doc
\usepackage{xcolor}%nicer colors, here used for links
\usepackage{dsfont}
\usepackage{stmaryrd}
\usepackage{geometry}
\usepackage{hyperref}
\usepackage{fancyhdr}
\usepackage{multicol}
\usepackage{csquotes}
\usepackage{enumitem}
\usepackage{pythonhighlight}

\usepackage[backend=biber,style=alphabetic,
citestyle=alphabetic]{biblatex} %biblatex mit biber laden
\addbibresource{sources.bib}

%settings colors for links
\hypersetup{
    colorlinks,
    linkcolor={blue!50!black},
    citecolor={blue},
    urlcolor={blue!80!black}
}

\definecolor{pblue}{rgb}{0.13,0.13,1}
\definecolor{pgreen}{rgb}{0,0.5,0}
\definecolor{pred}{rgb}{0.9,0,0}
\definecolor{pgrey}{rgb}{0.46,0.45,0.48}


\pagestyle{fancy}
\lhead{NW+VS -- Übung\\WiSe 2021/22}
\rhead{FB 4 -- IKG \\ HTW-Berlin}
\lfoot{Übungsblatt 02 -- Application Layer}
\cfoot{}
\fancyfoot[R]{\thepage}
\renewcommand{\headrulewidth}{0.4pt}
\renewcommand{\footrulewidth}{0.4pt}

\lstdefinestyle{Bash}{
  language=bash,
  showstringspaces=false,
  basicstyle=\small\sffamily,
  numbers=left,
  numberstyle=\tiny,
  numbersep=5pt,
  frame=trlb,
  columns=fullflexible,
  backgroundcolor=\color{gray!20},
  linewidth=0.9\linewidth,
  %xleftmargin=0.5\linewidth
}

% Python style for highlighting
\newcommand\pythonstyle{\lstset{
language=Python,
basicstyle=\ttm,
morekeywords={self},              % Add keywords here
keywordstyle=\ttb\color{deepblue},
emph={MyClass,__init__},          % Custom highlighting
emphstyle=\ttb\color{deepred},    % Custom highlighting style
stringstyle=\color{deepgreen},
frame=tb,                         % Any extra options here
showstringspaces=false
}}


%%here begins the actual document%%
\newcommand{\horrule}[1]{\rule{\linewidth}{#1}} % Create horizontal rule command with 1 argument of height

\begin{document}
\begin{center}
\Large{\textbf{Übungsblatt 02 --  Application Layer}}
\end{center}

\begin{center}\Large{\textbf{Aufgabe A -- Scapy}}\end{center}\vskip0.25in
Eine gute Anlaufstelle für die Benutzung von \emph{Scapy} finden sie unter: \url{https://scapy.readthedocs.io/en/latest/introduction.html}.
\begin{enumerate}
	\item Starten sie eine interaktive \emph{Scapy}-Session mithilfe des Terminals:
	\begin{lstlisting}[style=Bash, language=Bash]
scapy
	\end{lstlisting}
	Die funktioniert analog zur \emph{ipython} Shell.
	\begin{enumerate}
		\item Welche \emph{HTTP}-Protokolle und Funktionalitäten unterstützt \emph{Scapy}?
		\begin{python}
>>> explore(scapy.layers.http)
\end{python}
		\item Lassen sie in \emph{Scapy} zeigen, wie eine Default-HTTP-Request aussieht.
		\item Lassen sie in \emph{Scapy} zeigen, wie eine Default-HTTP-Response aussieht.
	\end{enumerate}		
	\item Wie können sie einzelne Layer (also Protokolle) laden?
	\item Laden sie den \emph{http}-Layer.
	\item Lesen sie folgenden \emph{HTTP}-Request:
	\begin{python}
load_layer("http")
req = HTTP()/HTTPRequest(
    Accept_Encoding=b'gzip, deflate',
    Cache_Control=b'no-cache',
    Connection=b'keep-alive',
    Host=b'HERE.WEB.URL',
    Pragma=b'no-cache'
)
a = TCP_client.tcplink(HTTP, "HERE.WEB.URL", PORT)
answser = a.sr1(req)
a.close()
with open("HERE.WEB.URL", "wb") as file:
    file.write(answser.load)
\end{python}
	\item Passen sie den Code so an, dass sie eine \emph{HTTP}-Anfrage an den Webserver der HTW stellen können.
\end{enumerate}

\begin{center}\Large{\textbf{Aufgabe B -- Chat-Server}}\end{center}\vskip0.25in

\begin{enumerate}
	\item unter folgender \emph{URL} finden sie den Code für den Chat-Server:
	\begin{enumerate}
		\item Laden sie den Code mittels \emph{wget} herunter. Lassen sie parallel Wireshark laufen und untersuchen sie den Ablauf.
		\item Öffnen sie den Server-Code und passen sie die stellen an, sodass der Server auf ihrer VM lauffähig ist.
		\item Welche IP-Adresse können sie nutzen und wie bringen sie diese in Erfahrung?
		\item Welchen Port können sie dem Server geben? Für welche Ports benötigen sie spezielle Rechte und für welche nicht?
		\item Geben sie dem Server entsprechende Rechte, um auf der VM ausführbar zu sein.\\
		Welches Recht benötigt der Server um ausführbar zu sein.
		\item Testen sie anschließend, ob ihr Server tatsächlich funktioniert.
	\end{enumerate}
	\item unter folgender \emph{URL} finden sie den Code für den Chat-Client:
	\begin{enumerate}
		\item Laden sie den Code mittels \emph{cURL} herunter. Lassen sie parallel Wireshark laufen und untersuchen sie den Ablauf.
		\item Öffnen sie den Client-Code und passen sie die stellen an, sodass der Server auf ihrer VM lauffähig ist.
		\item Welche IP-Adresse können sie nutzen und wie bringen sie diese in Erfahrung?
		\item Welchen Port können sie dem Client geben? Für welche Ports benötigen sie spezielle Rechte und für welche nicht?
		\item Geben sie dem Client entsprechende Rechte, um auf der VM ausführbar zu sein.\\
		Welches Recht benötigt der Server um ausführbar zu sein.
		\item Testen sie anschließend, ob ihr Server tatsächlich funktioniert.
	\end{enumerate}
	\item Wenn beide Applikationen laufen: Können sie den Server einer anderen VM erreichen? 
	\item Können sich innerhalb ihrer VM mehrere Clients auf die VM \enquote{connecten}?
\end{enumerate}

\end{document}
