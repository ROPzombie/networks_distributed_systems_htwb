%%%%%%%%%%%%%%%%%%%%%%%%%%%%%%%%%%%%%%%%%%%%%%%%%%%%%%%%%%%%%%%%%%%%%%%%%%
%%LaTeX template for papers && theses									%%
%%Done by the incredible ||Z01db3rg||									%%
%%Under the do what ever you want license								%%
%%%%%%%%%%%%%%%%%%%%%%%%%%%%%%%%%%%%%%%%%%%%%%%%%%%%%%%%%%%%%%%%%%%%%%%%%% 

%start preamble
\documentclass[paper=a4,fontsize=11pt]{scrartcl}%kind of doc, font size, paper size
\usepackage[ngerman]{babel}%for special german letters etc			
%\usepackage{t1enc} obsolete, but some day we go back in time and could use this again
\usepackage[T1]{fontenc}%same as t1enc but better						
\usepackage[utf8]{inputenc}%utf-8 encoding, other systems could use others encoding
%\usepackage[latin9]{inputenc}			
\usepackage{amsmath}%get math done
\usepackage{amsthm}%get theorems and proofs done
\usepackage{graphicx}%get pictures & graphics done
\graphicspath{{pictures/}}%folder to stash all kind of pictures etc
\usepackage{amssymb}%symbolics for math
\usepackage{amsfonts}%extra fonts
\usepackage []{natbib}%citation style
\usepackage{caption}%captions under everything
\usepackage{listings}
\usepackage[titletoc]{appendix}
\numberwithin{equation}{section} 
\usepackage[printonlyused,withpage]{acronym}%how to handle acronyms
\usepackage{float}%for garphics and how to let them floating around in the doc
\usepackage{cclicenses}%license!
\usepackage{xcolor}%nicer colors, here used for links
\usepackage{wrapfig}%making graphics floated by text and not done by minipage
\usepackage{dsfont}
\usepackage{stmaryrd}
\usepackage{geometry}
\usepackage{hyperref}
\usepackage{fancyhdr}
\usepackage{menukeys}

%settings colors for links
\hypersetup{
    colorlinks,
    linkcolor={blue!50!black},
    citecolor={blue},
    urlcolor={blue!80!black}
}

\definecolor{pblue}{rgb}{0.13,0.13,1}
\definecolor{pgreen}{rgb}{0,0.5,0}
\definecolor{pred}{rgb}{0.9,0,0}
\definecolor{pgrey}{rgb}{0.46,0.45,0.48}

\pagestyle{fancy}
\lhead{Benjamin Tröster\\Netzwerke Übung (SoSe18)}
\rhead{FB 4 -- Angewandte Informatik\\ HTW-Berlin}
\lfoot{Übungsblatt 05 -- Wireshark}
\cfoot{}
\fancyfoot[R]{\thepage}
\renewcommand{\headrulewidth}{0.4pt}
\renewcommand{\footrulewidth}{0.4pt}

\lstdefinestyle{Bash}{
  language=bash,
  showstringspaces=false,
  basicstyle=\small\sffamily,
  numbers=left,
  numberstyle=\tiny,
  numbersep=5pt,
  frame=trlb,
  columns=fullflexible,
  backgroundcolor=\color{gray!20},
  linewidth=0.9\linewidth,
  %xleftmargin=0.5\linewidth
}

\newlength\labelwd
\settowidth\labelwd{\bfseries viii.)}
\usepackage{tasks}
\settasks{counter-format =tsk[a].), label-format=\bfseries, label-offset=3em, label-align=right, label-width
=\labelwd, before-skip =\smallskipamount, after-item-skip=0pt}
\usepackage[inline]{enumitem}
\setlist[enumerate]{% (
labelindent = 0pt, leftmargin=*, itemsep=12pt, label={\textbf{\arabic*.)}}}

\pdfpkresolution=2400%higher resolution

%%here begins the actual document%%
\newcommand{\horrule}[1]{\rule{\linewidth}{#1}} % Create horizontal rule command with 1 argument of height

\DeclareMathOperator{\id}{id}

\begin{document}
\begin{center}
\Large{\textbf{Übungsblatt 6 -- Wireshark}}
\end{center}
\begin{center}\Large{\textbf{Aufgabe A -- Wireshark}}\end{center}\vskip0.25in

\begin{center}\Large{\textbf{Aufgabe B -- ARP}}\end{center}\vskip0.25in
Möglicherweise ist Ihnen aufgefallen, dass Ihr Netzwerk in der Planung zwar IP-Adressen nutzt, aber kein Router Verwendung findet. Der verwendete Switch ist ein OSI-Layer 2 Gerät und kommt ohne IP-Adressen zurecht. Ihre Raspberry Pis verlangen jedoch zwingend eine IP-Adresse von Ihnen. Um den Knoten ein wenig zu lösen, schauen wir uns das ARP Protokoll an. 
\begin{enumerate}
		\item Wie adressiert ein Switch die Pakte zwischen den Endknoten (also den Raspberry Pis)? \textbf{Hinweis:} Wie oben bereits erwähnt geschieht dies nicht mittels IP-Adressen.
		\item Wie sieht dieses Adressschema aus?
		\item Nennen Sie einige Protokolle der Sicherungsschicht (Data Link Layer).
		\item Das \emph{Address Resolution Protocol (ARP)} dient im Netzwerk dazu IP-Adressen der Vermittlungsschicht (Network Layer) den physische MAC-Adressen der Netzgeräte der Sicherungsschicht zuzuordnen. Recherchieren Sie die groben Funktionsweise des \emph{Address Resolution Protocols}.
		\item Da unser Uplink (Gateway des Labors) \glqq nur\grqq\ das alte IPv4 spricht ist ARP notwendig. Unter IPv6 gibt es kein ARP, wie wird dies dort gehandhabt?
\end{enumerate}
\end{document}
